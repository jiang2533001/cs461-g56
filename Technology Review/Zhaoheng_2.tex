\section{Programming language for achieving Database functionality}
        There are many options for programming language such as java, python, php. Each programming language will have different features. Our goal for this part is to figure out the suitable language for our product and avoid using many different languages. Because the more languages we use the more mistake we will might have.we generate several criteria for evaluating different language such as APIs, testability,security and tool support .

        \begin{table}[ht]
        \resizebox{\textwidth}{!}{\begin{tabular}{|c|c|c|c|c|c|c|}
            \hline
            \textbf{language} & \textbf{API for inserting } & \textbf{API for updating} & \textbf{API for listing table} & \textbf{Testability} & \textbf{Security} &\textbf{DynamoDB support}\\
            \hline
            Java & Yes & Yes & Yes & Testable & Secure & Yes\\
            \hline
            php & Yes & Yes & Yes & Testable & Normal & Yes\\
            \hline
            python & No & No & No & Testable & Secure & Yes\\
            \hline
        \end{tabular}}
        \end{table}
                
        \noindent Java is the optimal choice for achieving Database functionality. Here are the reason as following. The document of Amazon DynamoDB provide the APIs for basic functionality in Java such as inserting data, updating data and listing table. These APIs make the database functionality achieving more easily. Besides, Java is testable and more secure than the php. In amazon web service, AWS Java SDK will be a good option for achieving the DynamoDB table functionalities such as updating.  it is similar with the command line prompt in Windows. We could use it to do different implementations for the table.For instance, if we need to update some items in the table, we could use the APIs in AWS Java SDK to do the updating.However, AWS Java SDK is not efficient for uploading the items.As a result, we will only use it for simple operations such as creating table, deleting table, updating table.Based on the reasons list above, Java might be the suitable language for achieving different functionalities.
