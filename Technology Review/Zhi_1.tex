\section{The framework and storage of processing unprocessed data}
        In the entire large system, the client will collect data for us firstly, afterward we should do some analyzing and parsing for these unprocessed data. Meanwhile, we need to provide enough space to store these unprocessed data. In this step, the primary we must consider is to find a proper framework to build storage and then perform more operation like parsing data, so we will discuss framework and storage in this section.\\

        \noindent According to client’s requirement, we should use Amazon Web Service to complete these tasks. Although we have talked about advantages of EMR above, we also would like to choose Amazon EMR (Elastic MapReduce) in this part and we focus on Hadoop framework of Amazon ERM. Specifically, Hadoop is software framework that perform distributed processing a large amount of data across hundreds of inexpensive servers\cite{Z1}. The framework contains two kinds of tool. One is storage, and which is used store unprocessed data due to data cannot be stored in database directly. Another tool is used parse data. The advantage of Hadoop is obvious, because “Hadoop can provide a high level of durability and availability while still being able to process computational analytical workloads in parallel. The combination of availability, durability, and scalability of processing makes Hadoop a natural fit for big data workloads”\cite{Z2}. In Hadoop, there are many kinds of tools, so according to our researches, we find two effective tools which are used to store unprocessed data as a storage, and they can also interact with Hadoop framework.\\ 

        \noindent First of all, the Amazon Simple Storage Service (Amazon S3) is one of best choices for us. Our main part of product database is built on AWS cloud platform, thus one advantage of S3 is that it has high interactivity with our database. In the other words, it is effortless to build connection and transform data among them. In addition, according to client’s description, our product should be able deal with many kinds of data such as log file and stream, hence S3 is scalable and it can satisfy as much needs as our data. The cost of entire product is also an important criterion we need to consider, because client wants to compare cost between cloud product and local hardware, so one crucial benefit of choosing this service is its cost is low.\\

        \noindent Second technology about storage we find is Hadoop Distributed File System (HDFS). Compare with S3, the scalability of HDFS is not better than the former. The main difference is that HDFS depends on local storage, so it has to add larger hard drives or more machines to the cluster when it needs to expand storage space for more data\cite{Z3}. Meanwhile this weakness will cause that cost of it will be increased obviously. As for size limitation, any size of files can be allowed to store in HDFS, but the maximum size of single data element is only up to 5GB. We have not yet known all information about sample data client will provide, thus it is unclear whether this limitation of size on S3 will affect our product. Overall, S3 is better than HDFS because it can maximally decrease cost and ensure effective connection with database. 
